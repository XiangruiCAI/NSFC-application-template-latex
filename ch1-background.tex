\subsection{\sihao \kaishu \bfseries 1.1 研究背景与科学意义}
% 项目名称:面向动态异质网络的多粒度异常检测研究

异质网络(Heterogeneous Network),又称异质信息网络,是一种节点和边类型多样,用来描述现实世界中实体和实体间关系的数据结构。
在异质网络中,节点和边通常有多种类型,分别表示不同类型的实体和实体间关系,如在网上购物平台中,用户和商品是两种不同类型的节点,点击、收藏、购买等行为可以用不同类型的实体间关系来表达。
与单一类型的同质网络相比,异质网络具有更强的表达能力,能更好地描述现实世界,已经广泛用于推荐系统、社交网络、生物信息学等领域。
为进一步建立对异质网络时序演化的认知,动态异质网络应运而生。
在异质网络的基础上,动态异质网络为节点和边增加了时间属性,不同时刻节点和边的类型可能不同,网络结构也会随时间演化。
因此,动态异质网络可以更好地刻画现实数据场景,如用户行为的持续建模、事件演化分析和患者病情进展建模等。

异常检测一直是学术界和工业界的研究热点和难点,
国内外高校(清华大学、上海交通大学,新加坡国立大学、普渡大学、卡内基梅隆大学)和研究机构(中国科学院自动化研究所、IBM、亚马逊、华为、阿里巴巴),均致力于异常检测研究。
2023年清华大学、南开大学和中国计算机学会联合举办国际智能运维挑战赛,\footnote{\url{http://aiops-challenge.com}}参赛队伍基于多源异构数据进行异常检测、故障定位和故障分类等。
%2021年SIGKDD 会议联合加州大学尔湾分校举办了多源时序数据异常检测挑战赛(KDD Cup),\footnote{\url{https://compete.hexagon-ml.com/practice/competition/39/}}上海深兰科技和华为诺亚方舟实验室分获冠亚军。
2022年亚马逊和多所国际知名高校在SIGKDD会议组织题为“异常和新颖性检测、解释和调节”研讨会(ANDEA),\footnote{\url{https://sites.google.com/view/andea2022/}},关注异常和新颖性检测的新技术、模型解释和改进方法。
随着信息系统的发展,对用户及其行为的描述形式日益多样化。
动态异质网络从结构、语义和时序演化三方面刻画现实数据场景,有助于建模各类信息系统的多源异构用户数据。
面向动态异质网络的异常检测旨在从动态异质网络中识别出异常或者具有潜在风险的节点或边,对于许多现实世界的复杂系统具有重要的应用价值,如社交网络、金融交易网络等,可以帮助发现动态异质网络中的异常事件、潜在的风险因素或者突变模式,从而对系统的安全性、稳定性和健康状态进行监测和管理。


% 当前异常检测的现状和不足
% 1. 多源异构数据缺乏统一表征
% 2. 单点检测准确率低,false positive高
% 3. 告警间关系不清,归因困难
综上可见,针对用户行为数据的异常检测已经成为学术界的研究热点,也收到工业界的广泛关注。


% 多源异构数据融合、时序图的优势

% 本项目的意义
\subsection{\sihao \songti \bfseries 1.2 国内外相关工作}


