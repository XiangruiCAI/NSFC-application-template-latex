\subsection{\sihao \kaishu \bfseries 1.1 研究背景与科学意义}
% 项目名称:多源异构数据融合的多粒度序列异常检测研究

行为序列记录一系列按照时间顺序发生的用户行为,从大规模行为序列中识别异常行为具有重要的应用价值,得到了学术界和工业界的广泛关注。
针对人物行为序列的异常检测可以及时发现异常行为,防范危害公共安全事件;抽取网络和系统日志中的用户行为序列并进行异常检测可以支撑网络入侵检测和内部威胁检测等安全应用;在电子商务平台中分析用户行为序列可以挖掘用户异常交易行为,保障数字经济安全。
然而,\textbf{\songti 随着网络和信息系统的发展,用户行为描述形式日益丰富和多样化,仅仅分析用户行为序列,无法准确描述行为语义,难以识别真实的异常行为和用户,也无法得到异常推断的原因。}

% 基于多源异构数据的异常检测举例,前沿情况、数据支撑
序列异常检测需要分析多源异构数据。

% 在公共安全领域,事件序列用于描述人物行为,包括其社交媒体言论、购物记录、交通信息等,人们的日常生活通过这些多源异构数据得到了全面的记录,检测人物异常行为可有效防范;在网络安全领域,事件序列用于描述用户的行为,包括收发邮件、操作文件、访问服务器等,用户的网络行为被多源异构日志数据全面记录,针对网络用户行为的异常检测可支持入侵检测和内部威胁检测等。

% 当前异常检测的现状和不足
% 1. 多源异构数据缺乏统一表征范式
% 2. 单点检测准确率低,false postive高
% 3. 告警间关系不清,归因困难

% 多源异构数据融合、时序图的优势

% 本项目的意义
\subsection{\sihao \songti \bfseries 1.2 国内外相关工作}


