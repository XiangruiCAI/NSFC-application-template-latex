\subsection{\sihao \kaishu \bfseries 1.1 研究背景与科学意义}
% 项目名称:多源异构数据融合的多粒度序列异常检测研究

行为序列记录一系列按照时间顺序发生的用户行为,从大规模行为序列中识别异常行为具有重要的应用价值,得到了学术界和工业界的广泛关注。
针对人物行为序列的异常检测可以及时发现异常行为,防范危害公共安全事件;抽取网络和系统日志中的用户行为序列并进行异常检测可以支撑网络入侵检测和内部威胁检测等安全应用;在电子商务平台中分析用户行为序列可以挖掘用户异常交易行为,保障数字经济安全。
然而,\textbf{\songti 随着网络和信息系统的发展,用户行为描述形式日益丰富和多样化,仅仅分析用户行为序列,无法准确描述行为语义,难以识别真实的异常行为和用户,也无法得到异常推断的原因。}

% 基于多源异构数据的异常检测举例,前沿情况、数据支撑
以内部威胁检测为例,图~\ref{fig:example}展示了某用户离职前违规盗取原单位项目源代码和用户数据,为实现改目标,该用户产生多个行为,被分散记录在不同的日志中,
包括网络日志、操作系统日志、数据库日志和应用日志,以及防火墙和杀毒软件日志,这些日志的结构各不相同,用户行为的抽取方法和描述方式也不相同。
仅分析单一来源的日志无法形成对用户行为和意图的整体认知,难以准确检测用户的异常行为和异常用户。
为准确检测该用户的异常行为,必须融合上述多源异构日志数据行分析。
类似地,面向公共安全事件的人物异常行为检测需要对关键人物在多媒体平台上的数据进行综合分析研判,包括其社交网络言论、交通信息、购物记录等多源异构数据;而在电子商务平台中,需要融合用户交易记录、历史评价、好友关系、关注和收藏等信息,才能综合分析用户意图,准确识别异常行为。
因此,面向用户行为序列的异常检测需要分析多源异构用户行为数据。

\begin{figure}[htp]
    \centering
    \includegraphics*[width=0.9\linewidth]{Figure1-Example.pdf}
    \caption{多源异构数据融合的序列异常检测示例。}
    \label{fig:example}
\end{figure}

% 在公共安全领域,事件序列用于描述人物行为,包括其社交媒体言论、购物记录、交通信息等,人们的日常生活通过这些多源异构数据得到了全面的记录,检测人物异常行为可有效防范;在网络安全领域,事件序列用于描述用户的行为,包括收发邮件、操作文件、访问服务器等,用户的网络行为被多源异构日志数据全面记录,针对网络用户行为的异常检测可支持入侵检测和内部威胁检测等。

% 当前异常检测的现状和不足
% 1. 多源异构数据缺乏统一表征
% 2. 单点检测准确率低,false positive高
% 3. 告警间关系不清,归因困难

% 多源异构数据融合、时序图的优势

% 本项目的意义
\subsection{\sihao \songti \bfseries 1.2 国内外相关工作}


