%!TEX program = xelatex
% 编译顺序: xelatex -> bibtex -> xelatex -> xelatex
% 国家自然科学基金NSFC面上项目申请书正文模板(2023年版)version1.0
% 声明:
% 注意!!!非国家自然科学基金委官方模版!!!由个人根据官方MsWord模版制作。本模版的作者尽力使本模版和官方模版生成的PDF文件视觉效果大致一样,然而,并不保证本模版有用,也不对使用本模版造成的任何直接或间接后果负责。 不得将本模版用于商用或获取经济利益。本模版可以自由修改以满足用户自己的需要。但是如果要传播本模版,则只能传播未经修改的版本。使用本模版意味着同意上述声明。
% 强烈建议自己对照官方MsWord模板确认格式和文字是否一致,尤其是蓝字部分。
% 如有问题,可以发邮件到ryanzz@foxmail.com



\documentclass[12pt,UTF8,AutoFakeBold=2,a4paper]{ctexart} %默认小四号字。允许楷体粗体。
\usepackage[english]{babel} %支持混合语言
\usepackage[dvipsnames]{xcolor}
\usepackage{graphicx} 
\usepackage{amsmath} %更多数学符号
\usepackage{wasysym}
\usepackage[unicode]{hyperref} %提供跳转链接
\usepackage{geometry} %改改尺寸
\usepackage{gbt7714}
\usepackage{natbib}
\setlength{\bibsep}{0.0pt}

%\geometry{left=3.23cm,right=3.23cm,top=2.54cm,bottom=2.54cm}
%latex的页边距比word的视觉效果要大一些,稍微调整一下
%\geometry{left=2.95cm,right=2.95cm,top=2.54cm,bottom=2.54cm}%2020
%\geometry{left=2.95cm,right=2.95cm,top=2.54cm,bottom=2.54cm}
\geometry{left=3.00cm,right=3.07cm,top=2.67cm,bottom=3.27cm}
\pagestyle{empty}
\setcounter{secnumdepth}{-2} %不让那些section和subsection自带标号,标号格式自己掌握
\definecolor{MsBlue}{RGB}{0,112,192} %Ms Word 的蓝色和latex xcolor包预定义的蓝色不一样。通过屏幕取色得到。
% Renaming floats with babel
\addto\captionsenglish{
    \renewcommand{\contentsname}{目录}
    \renewcommand{\listfigurename}{插图目录}
    \renewcommand{\listtablename}{表格}
    %\renewcommand{\refname}{\sihao 参考文献}
    \renewcommand{\refname}{\sihao \kaishu \leftline{参考文献}} %这几个字默认字号稍大,改成四号字,楷书,居左(默认居中) 根据喜好自行修改,官方模板未作要求
    \renewcommand{\abstractname}{摘要}
    \renewcommand{\indexname}{索引}
    \renewcommand{\tablename}{表}
    \renewcommand{\figurename}{图}
    } %把Figure改成‘图’,reference改成‘参考文献’。如此处理是为了避免和babel包冲突。
%定义字号
\newcommand{\chuhao}{\fontsize{42pt}{\baselineskip}\selectfont}
\newcommand{\xiaochuhao}{\fontsize{36pt}{\baselineskip}\selectfont}
\newcommand{\yihao}{\fontsize{26pt}{\baselineskip}\selectfont}
\newcommand{\erhao}{\fontsize{22pt}{\baselineskip}\selectfont}
\newcommand{\xiaoerhao}{\fontsize{18pt}{\baselineskip}\selectfont}
\newcommand{\sanhao}{\fontsize{16pt}{\baselineskip}\selectfont}
\newcommand{\sihao}{\fontsize{14pt}{\baselineskip}\selectfont}
\newcommand{\xiaosihao}{\fontsize{12pt}{\baselineskip}\selectfont}
\newcommand{\wuhao}{\fontsize{10.5pt}{\baselineskip}\selectfont}
\newcommand{\xiaowuhao}{\fontsize{9pt}{\baselineskip}\selectfont}
\newcommand{\liuhao}{\fontsize{7.875pt}{\baselineskip}\selectfont}
\newcommand{\qihao}{\fontsize{5.25pt}{\baselineskip}\selectfont}
%字号对照表
%二号 21pt
%四号 14
%小四 12
%五号 10.5
%设置行距 1.5倍
\renewcommand{\baselinestretch}{1.5}
%\XeTeXlinebreaklocale "zh"           % 中文断行

%%%% 正文开始 %%%%
\begin{document}
\begin{center}
{\sanhao \kaishu \bfseries 报告正文}
\end{center}

{\sihao \kaishu 参照以下提纲撰写,要求内容翔实、清晰,层次分明,标题突出。{\color{MsBlue} \bfseries 请勿删除或改动下述提纲标题及括号中的文字。}}
\vskip -5mm
{\color{MsBlue} \subsection{\sihao \kaishu \quad \ (一)立项依据与研究内容(建议8000字以下): }}

{\sihao \kaishu \color{MsBlue} 1.{\bfseries 项目的立项依据}(研究意义、国内外研究现状及发展动态分析,需结合科学研究发展趋势来论述科学意义;或结合国民经济和社会发展中迫切需要解决的关键科技问题来论述其应用前景。附主要参考文献目录);}

\subsection{\sihao \kaishu \bfseries 1.1 研究背景与科学意义}
% 项目名称:面向动态异质网络的多粒度异常检测研究

异质网络(Heterogeneous Network),又称异质信息网络,是一种节点和边类型多样,用来描述现实世界中实体和实体间关系的数据结构。
在异质网络中,节点和边通常有多种类型,分别表示不同类型的实体和实体间关系,如在网上购物平台中,用户和商品是两种不同类型的节点,点击、收藏、购买等行为可以用不同类型的实体间关系来表达。
与单一类型的同质网络相比,异质网络具有更强的表达能力,能更好地描述现实世界,已经广泛用于推荐系统、社交网络、生物信息学等领域。
为进一步建立对异质网络时序演化的认知,动态异质网络应运而生。
在异质网络的基础上,动态异质网络为节点和边增加了时间属性,不同时刻节点和边的类型可能不同,网络结构也会随时间演化。
因此,动态异质网络可以更好地刻画现实数据场景,如用户行为的持续建模、事件演化分析和患者病情进展建模等。

异常检测一直是学术界和工业界的研究热点和难点,
国内外高校(清华大学、上海交通大学,新加坡国立大学、普渡大学、卡内基梅隆大学)和研究机构(中国科学院自动化研究所、IBM、亚马逊、华为、阿里巴巴),均致力于异常检测研究。
2023年清华大学、南开大学和中国计算机学会联合举办国际智能运维挑战赛,\footnote{\url{http://aiops-challenge.com}}参赛队伍基于多源异构数据进行异常检测、故障定位和故障分类等。
%2021年SIGKDD 会议联合加州大学尔湾分校举办了多源时序数据异常检测挑战赛(KDD Cup),\footnote{\url{https://compete.hexagon-ml.com/practice/competition/39/}}上海深兰科技和华为诺亚方舟实验室分获冠亚军。
2022年亚马逊和多所国际知名高校在SIGKDD会议组织题为“异常和新颖性检测、解释和调节”研讨会(ANDEA),\footnote{\url{https://sites.google.com/view/andea2022/}},关注异常和新颖性检测的新技术、模型解释和改进方法。
随着信息系统的发展,对用户及其行为的描述形式日益多样化。
动态异质网络从结构、语义和时序演化三方面刻画现实数据场景,有助于建模各类信息系统的多源异构用户数据。
面向动态异质网络的异常检测旨在从动态异质网络中识别出异常或者具有潜在风险的节点或边,对于许多现实世界的复杂系统具有重要的应用价值,如社交网络、金融交易网络等,可以帮助发现动态异质网络中的异常事件、潜在的风险因素或者突变模式,从而对系统的安全性、稳定性和健康状态进行监测和管理。


% 当前异常检测的现状和不足
% 1. 多源异构数据缺乏统一表征
% 2. 单点检测准确率低,false positive高
% 3. 告警间关系不清,归因困难
综上可见,针对用户行为数据的异常检测已经成为学术界的研究热点,也收到工业界的广泛关注。


% 多源异构数据融合、时序图的优势

% 本项目的意义
\subsection{\sihao \songti \bfseries 1.2 国内外相关工作}




\bibliographystyle{gbt7714-numerical}
\bibliography{references}
\newpage

{\sihao \color{MsBlue} \kaishu 2. {\bfseries 项目的研究内容、研究目标,以及拟解决的关键科学问题}(此部分为重点阐述内容);}



{\sihao \color{MsBlue} \kaishu 3.{\bfseries 拟采取的研究方案及可行性分析} (包括研究方法、技术路线、实验手段、关键技术等说明);}


{\sihao \color{MsBlue} \kaishu 4.{\bfseries 本项目的特色与创新之处;}}



{\sihao \color{MsBlue} \kaishu 5.{\bfseries 年度研究计划及预期研究结果}(包括拟组织的重要学术交流活动、国际合作与交流计划等)。}



{\color{MsBlue} \subsection{\sihao \kaishu \quad \ (二)研究基础与工作条件 }}

{\sihao \color{MsBlue} \kaishu 1.{\bfseries 研究基础}(与本项目相关的研究工作积累和已取得的研究工作成绩);}


{\sihao \color{MsBlue} \kaishu 2.{\bfseries 工作条件}(包括已具备的实验条件,尚缺少的实验条件和拟解决的途径,包括利用国家实验室、国家重点实验室和部门重点实验室等研究基地的计划与落实情况);}


{\sihao \color{MsBlue} \kaishu 3.{\bfseries 正在承担的与本项目相关的科研项目情况}(申请人和主要参与者正在承担的与本项目相关的科研项目情况,包括国家自然科学基金的项目和国家其他科技计划项目,要注明项目的资助机构、项目类别、批准号、项目名称、获资助金额、起止年月、与本项目的关系及负责的内容等);}

无。

{\sihao \color{MsBlue} \kaishu 4.{\bfseries 完成国家自然科学基金项目情况}(对申请人负责的前一个已资助期满的科学基金项目(项目名称及批准号)完成情况、后续研究进展及与本申请项目的关系加以详细说明。另附该项目的研究工作总结摘要(限500字)和相关成果详细目录)。}


{\color{MsBlue} \subsection{\sihao \kaishu \quad \ (三)其他需要说明的情况 }}

{\sihao \color{MsBlue} \kaishu 1. 申请人同年申请不同类型的国家自然科学基金项目情况(列明同年申请的其他项目的项目类型、项目名称信息,并说明与本项目之间的区别与联系)。 }

无。

{\sihao \color{MsBlue} \kaishu 2. 具有高级专业技术职务(职称)的申请人或者主要参与者是否存在同年申请或者参与申请国家自然科学基金项目的单位不一致的情况;如存在上述情况,列明所涉及人员的姓名,申请或参与申请的其他项目的项目类型、项目名称、单位名称、上述人员在该项目中是申请人还是参与者,并说明单位不一致原因。}

无。

{\sihao \color{MsBlue} \kaishu 3. 具有高级专业技术职务(职称)的申请人或者主要参与者是否存在与正在承担的国家自然科学基金项目的单位不一致的情况;如存在上述情况,列明所涉及人员的姓名,正在承担项目的批准号、项目类型、项目名称、单位名称、起止年月,并说明单位不一致原因。}

无。

{\sihao \color{MsBlue} \kaishu 4. 其他。}

无。
\end{document}
