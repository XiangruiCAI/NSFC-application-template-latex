\subsection{\sihao \kaishu \bfseries 1.1 研究背景与科学意义}
% 项目名称:多源异构数据融合的多粒度序列异常检测研究
% 项目名称:面向动态异质网络的多粒度异常检测研究

行为序列记录一系列按照时间顺序发生的用户行为,从大规模行为序列中识别异常行为具有重要的应用价值,得到了学术界和工业界的广泛关注。
针对人物行为序列的异常检测可以及时发现异常行为,防范危害公共安全事件;抽取网络和系统日志中的用户行为序列并进行异常检测可以支撑网络入侵检测和内部威胁检测等安全应用;在电子商务平台中分析用户行为序列可以挖掘用户异常交易行为,保障数字经济安全。
然而,\textbf{\songti 随着网络和信息系统的发展,用户行为描述形式日益丰富和多样化,仅仅分析用户行为序列,无法准确描述行为语义,难以识别真实的异常行为和用户,也无法得到异常推断的原因。}

% 基于多源异构数据的异常检测举例,前沿情况、数据支撑
以内部威胁检测为例,图~\ref{fig:example}展示了某用户离职前违规盗取原单位项目源代码和用户数据,为实现改目标,该用户产生多个行为,被分散记录在不同的日志中,
包括网络日志、操作系统日志、数据库日志和应用日志,以及防火墙和杀毒软件日志,这些日志的结构各不相同,用户行为的抽取方法和描述方式也不相同。
仅分析单一来源的日志无法形成对用户行为和意图的整体认知,难以准确检测用户的异常行为和异常用户。
为准确检测该用户的异常行为,必须融合上述多源异构日志数据行分析。
类似地,面向公共安全事件的人物异常行为检测需要对关键人物在多媒体平台上的数据进行综合分析研判,包括其社交网络言论、交通信息、购物记录等多源异构数据;而在电子商务平台中,需要融合用户交易记录、历史评价、好友关系、关注和收藏等信息,才能综合分析用户意图,准确识别异常行为。
因此,面向用户行为序列的异常检测需要分析多源异构用户行为数据。

\begin{figure}[htp]
    \centering
    \includegraphics*[width=0.9\linewidth]{Figure1-Example.pdf}
    \caption{多源异构数据融合的序列异常检测示例。}
    \label{fig:example}
\end{figure}

% 当前异常检测的现状和不足
% 1. 多源异构数据缺乏统一表征
% 2. 单点检测准确率低,false positive高
% 3. 告警间关系不清,归因困难
用户行为序列广泛应用于各类信息系统,异常检测可以及时发现恶意行为,有助于规避系统的潜在风险,得到了国内外学术界和工业界的广泛关注。
各大高校(清华大学、上海交通大学,新加坡国立大学、普渡大学、卡内基梅隆大学)、研究机构(中国科学院自动化研究所、IBM、亚马逊、华为、阿里巴巴),均致力于序列异常检测研究,以支撑各类系统和应用稳定、安全地运行。
2021年SIGKDD 会议联合加州大学尔湾分校举办了多源时序数据异常检测挑战赛(KDD Cup)\footnote{\url{https://compete.hexagon-ml.com/practice/competition/39/}},上海深兰科技和华为诺亚方舟实验室分获冠亚军。2022年亚马逊和多所国际知名高校在SIGKDD会议组织题为“异常和新颖性检测、解释和调节”研讨会,关注异常和新颖性检测的新技术、模型解释和改进方法。ODD 2021 workshop
综上可见,针对用户行为数据的异常检测已经成为学术界的研究热点,也收到工业界的广泛关注。

全面准确地建模用户行为序列,理解用户意图,是及时检测异常行为的前提。

% 多源异构数据融合、时序图的优势

% 本项目的意义
\subsection{\sihao \songti \bfseries 1.2 国内外相关工作}


